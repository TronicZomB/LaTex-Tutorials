\documentclass{article}

\usepackage{circuitikz}

\begin{document}
\section{Monopoles}
	\begin{figure}[h!]
	\begin{circuitikz}
		\draw (-1,0) to[short,o-o] (1,0);
	\end{circuitikz}
	\end{figure}

	\begin{figure}[h!]
	\begin{circuitikz}
		\draw (-1,0) to[short,*-] (1,0);
	\end{circuitikz}
	\end{figure}

	\begin{figure}[h!]
	\begin{circuitikz}
		\draw (-1,0) to[short,o-o] (1,0);
		\draw (0,0) to[short] node[ground] {} (0,-1);
	\end{circuitikz}
	\end{figure}

\section{Bipoles}
	\begin{figure}[h!]
	\begin{circuitikz}
		\draw (0,0) to[R,i=$i_1$] (2,0);
	\end{circuitikz}
	\end{figure}

	\begin{figure}[h!]
	\begin{circuitikz}
		\draw (0,0) to[R,v_<=$v_1$] (2,0);
	\end{circuitikz}
	\end{figure}

	\begin{figure}[h!]
	\begin{circuitikz}
		\draw (0,0) to[R,l_=$R_1$] (2,0);
	\end{circuitikz}
	\end{figure}

\section{Tripoles}
	\begin{figure}[h!]
	\begin{circuitikz}
		\draw (0,0) node[npn](npn1) {}
		(npn1.base) node[anchor=east] {B}
		(npn1.collector) node[anchor=south] {C}
		(npn1.emitter) node[anchor=north] {E};
	\end{circuitikz}
	\end{figure}	

	\begin{figure}[h!]
	\begin{circuitikz}
		\draw (0,0) node[npn](npn1) {}
		(npn1.base) node[anchor=east] {B}
		(npn1.collector) node[anchor=south,xshift=0.5cm] {C}
		(npn1.emitter) node[anchor=north] {E};
		\draw (npn1.collector) to[R] ++(0,2);
	\end{circuitikz}
	\end{figure}	

\end{document}
